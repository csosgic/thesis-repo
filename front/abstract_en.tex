\begin{abstractEn}
	

	The smart grid's reliability and security are at significant risk due to the increasing reliance on smart grid technologies, which have brought in new vulnerabilities which is cyber threats. Many traditional intrusion detection systems (IDSs) are unable to detect most of these threats because of the complexity and  of smart grid data.This thesis introduces a deep learning-based network intrusion detection system (DL-NIDS) that tackles this issue

	The DL-based NIDS uses an innovative architecture utilizing convolutional neural networks (CNNs) and recurrent neural networks (RNNs) to discover deviations in smart grid network traffic. This includes normal patterns as well as those that are maleficent.

	To estimate its performance, these models were trained on real life network traffic data then tested on a new set of data. In contrast to conventional IDSs, there is a dramatic improvement in both robustness and detection accuracy as shown by results. The system is also demonstrated to be capable of detecting a variety of DoS and DDoS attacks. Therefore, this research enhances the dependability and safety of such critical infrastructure systems by contributing towards better cybersecurity solutions for smart grids.


\end{abstractEn}

\begin{keywordsEn}
	(smart grid, cybersecurity, deep learning, denial of service attacks, Convolutional neural network , Long short-term memory)
\end{keywordsEn}



