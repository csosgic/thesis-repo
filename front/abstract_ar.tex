\begin{abstractAr}





		تتعرض موثوقية الشبكة الذكية وأمنها لخطر كبير بسبب الاعتماد المتزايد على تقنيات الشبكة الذكية، والتي جلبت نقاط ضعف جديدة تتمثل في التهديدات السيبرانية. العديد من أنظمة كشف التسلل التقليدية \LR{(IDSs)} غير قادرة على اكتشاف معظم هذه التهديدات بسبب تعقيد الشبكة الذكية. تقدم هذه الأطروحة نظام كشف التسلل إلى الشبكة القائم على التعلم العميق \LR{(DL-NIDS)} الذي يعالج هذه المشكلة
		
		  تستخدم \LR{NIDS} المستندة إلى \LR{DL} بنية مبتكرة تستخدم الشبكات العصبية التلافيفية \LR{(CNNs)} والشبكات العصبية المتكررة \LR{(RNNs)} لاكتشاف الانحرافات في حركة مرور شبكة الشبكة الذكية. وهذا يشمل الأنماط العادية وكذلك تلك المؤذية.
		
		  ولتقدير أدائها، تم تدريب هذه النماذج على بيانات حركة مرور الشبكة الحقيقية ثم تم اختبارها على مجموعة جديدة من البيانات. وعلى النقيض من أجهزة كشف الهوية التقليدية، هناك تحسن كبير في كل من المتانة ودقة الكشف كما هو موضح في النتائج. كما ثبت أن النظام قادر على اكتشاف مجموعة متنوعة من هجمات \LR{(DoS)} و \LR{(DDoS)}. ولذلك، فإن هذا البحث يعزز موثوقية وسلامة أنظمة البنية التحتية الحيوية هذه من خلال المساهمة في إيجاد حلول أفضل للأمن السيبراني للشبكات الذكية.
		
		

\end{abstractAr}

\begin{keywordsAr}
	(الشبكة الذكية، الأمن السيبراني، التعلم العميق، هجمات حجب الخدمة، الشبكة العصبية التلافيفية، الذاكرة الطويلة قصيرة المدى)
\end{keywordsAr}

\newpage