%\begin{abstractFr}
%	
%	(Vous insérez ici le résumé du manuscrit en français, dont le nombre de mots \textbf{ne doit pas dépasser 200}.)
%	
%\end{abstractFr}
%
%\begin{keywordsFr}
%	(6 mots clés au maximum doivent être séparés par des virgules ",")
%\end{keywordsFr}




\begin{abstractFr}


La fiabilité et la sécurité du réseau intelligent sont considérablement menacées en raison du recours croissant aux technologies de réseau intelligent, qui ont introduit de nouvelles vulnérabilités que sont les cybermenaces. De nombreux systèmes de détection d'intrusion (IDS) traditionnels sont incapables de détecter la plupart de ces menaces en raison de la complexité et des données des réseaux intelligents. Cette thèse présente un système de détection d'intrusion réseau basé sur l'apprentissage profond (DL-NIDS) qui s'attaque à ce problème.

  Le NIDS basé sur DL utilise une architecture innovante utilisant des réseaux neuronaux convolutifs (CNN) et des réseaux neuronaux récurrents (RNN) pour découvrir les écarts dans le trafic du réseau intelligent. Cela inclut les schémas normaux ainsi que ceux qui sont maléfiques.

  Pour estimer ses performances, ces modèles ont été formés sur des données réelles de trafic réseau puis testés sur un nouvel ensemble de données. Contrairement aux IDS conventionnels, les résultats montrent une amélioration considérable de la robustesse et de la précision de la détection. Le système s’est également révélé capable de détecter diverses attaques DoS et DDoS. Par conséquent, cette recherche améliore la fiabilité et la sécurité de ces systèmes d’infrastructures critiques en contribuant à de meilleures solutions de cybersécurité pour les réseaux intelligents.


  \end{abstractFr}



  \begin{keywordsEn}
	(grille intelligente, cybersécurité, apprentissage profond, attaques par déni de service, réseau neuronal convolutif, mémoire à long terme)
\end{keywordsEn}

\newpage
