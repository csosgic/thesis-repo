\documentclass{masterthesis-uc2-en}

\usepackage{listings}
\usepackage{color}
\usepackage{lipsum}
\usepackage[svgnames]{xcolor}
\usepackage[tikz]{bclogo}


%% Metadata
\department{Department of Fundamental Computing and its Applications (IFA)}
\specialty{Sciences and Technologies of Information and Communication (STIC)}
\title{Cybersecurity of smart grid infratsructure communication} 
\studentFullnameA{ochetati ilyes chiheb eddine}
\studentFullnameB{kechicheb ahmed}
\supervisorFullnameA{Salim benayoune}
\session{June 2024}

%% Additional commands
%-------------------------------------------------------
% Listings (Source codes) configuration
%-------------------------------------------------------

\definecolor{color-lst-string}{RGB}{166,95,18}
\definecolor{color-lst-annotation}{RGB}{128,128,0}
\definecolor{color-lst-constant}{RGB}{5,133,5}
\definecolor{color-lst-keyword}{RGB}{33,33,255}
\definecolor{color-lst-class}{RGB}{0,112,192}
\definecolor{color-lst-accessor}{RGB}{125,0,84}
\definecolor{color-lst-red}{RGB}{255,0,0}
\definecolor{color-lst-comment}{RGB}{168,166,171}

% XML
\lstdefinelanguage{XML}{
	alsoletter=-.,
	morestring=[b]",
	morecomment=[s]{<?}{?>},
	morecomment=[s]{<!--}{-->},
	morekeywords={
		manifest,application,activity,uses-permission,
	%	...
	},
}

% Java
\lstdefinelanguage{Java}{
	alsoletter=&,
	deletekeywords=SAVE,%	language=Java,
	morestring=[b]',
	morestring=[b]",
	morecomment=[s]{/*}{*/},
	morecomment=[l]{//},
	keywordstyle={[1]\color{color-lst-keyword}},
	morekeywords=[1]{
		super,import,class,interface,void,extends,implements,new,this,null,true,false,return,break,
		if,else,for,while,foreach,switch,case,
		int,long,float,double,boolean,&&,
	},
	keywordstyle={[2]\color{color-lst-class}},
	morekeywords=[2]{
		Bundle,String,Activity,R,View,TextView,Button,List,ArrayList,ViewGroup,Build,Manifest,
%		...
	},
	keywordstyle={[3]\color{color-lst-accessor}},
	morekeywords=[3]{public,private,protected,package},
	keywordstyle={[4]\color{color-lst-annotation}},
	morekeywords=[4]{@Override},
	keywordstyle={[5]\color{color-lst-constant}},
	morekeywords=[5]{
		LENGTH\_LONG,LENGTH\_SHORT,ACTION_CALL,RESULT_OK,
%		...
	},
}

% HTML

% Javascript

% ...

\begin{document}


\frontmatter
	\pagestyle{plain}
	
	\hypersetup{citecolor=blue}
	
	\definecolor{dkgreen}{rgb}{0,0.6,0}
	\definecolor{gray}{rgb}{0.5,0.5,0.5}
	\definecolor{mauve}{rgb}{0.58,0,0.82}
	
	\lstset{frame=tb,
	  language=c++,
	  aboveskip=3mm,
	  belowskip=3mm,
	  showstringspaces=false,
	  columns=flexible,
	  basicstyle={\small\ttfamily},
	  numbers=none,
	  numberstyle=\tiny\color{gray},
	  keywordstyle=\color{blue},
	  commentstyle=\color{dkgreen},
	  stringstyle=\color{mauve},
	  breaklines=true,
	  breakatwhitespace=true,
	  tabsize=3
	}
	





	\section{installation instrucions}
	
	
	\firmlist
	\begin{itemize}
		\item ubuntu 18.04 required (https://releases.ubuntu.com/18.04/)
		\item python3.6.9 installed by default should be enough
	\end{itemize}
	
	
	\begin{lstlisting}
		sudo apt update && sudo apt upgrade
		sudo apt install xterm
	\end{lstlisting}

	
	do all the 6 commands below as i am not sure which one it needs 

	\begin{lstlisting}
		apt install python3-pip
		pip3 install matplotlib graphviz pandas
		sudo apt install python3-tk	
	\end{lstlisting}


	\begin{lstlisting}
		apt install python-pip
		pip install matplotlib graphviz pandas
		sudo apt install python-tk
	\end{lstlisting}



	\subsection{installation FNCS, gridlab, ns3}
	follow along with the  \path{installation_guide.md} in the project directory except for 1 needed change
	when installing gridlab switch to the "feature/797" branch
	\begin{lstlisting}
	git checkout feature/797
	\end{lstlisting}



	\subsection{editing some files to get the simulation to run}



	we will be using (13 Nodes 73 Houses)

	changes:
	\subsubsection{attack borker python file}
	\path{attack_broker.py}
	on line 103  

	replace:
	\begin{lstlisting}
	pid = int(child.communicate()[0].split('\n')[0])
	\end{lstlisting}
	with:

	\begin{lstlisting}
		stdout_data, stderr_data = child.communicate()
		pid = int(stdout_data.split(b'\n')[0])
	\end{lstlisting}


	\subsubsection{gridlab glm file}
	\path{gridlab-D.glm}
	remove on line	75
	\begin{lstlisting}
	object fncs_msg {
     	name fncs_msg;
 	 		 parent Market_1;
     	route "function:controller/submit_bid_state -> auction/submit_bid_state";
     	option "transport:hostname localhost, port 5570";
     	configure fncs_msg.txt;
	}
	\end{lstlisting}




	
	\section{the implementation}

	\begin{itemize}
		\item the IDS function is in this same repository in \path{listings/IDS.cc}
		\item what the new \path{ns-3.cc} file in your project should be is \path{listings/new_ns-3.cc}
		\item so copy this file (\path{listings/new_ns-3.cc}) into your \path{ns-3.ns} to test it
	\end{itemize}


	\begin{lstlisting}
	\end{lstlisting}



	\section{conclusion about the simulation}

	the ns3 section gets stuck at the line     
	\begin{lstlisting}
		FncsSimulatorImpl *hb=new FncsSimulatorImpl();
	\end{lstlisting}
	tried removing that section but nothing diffrent happend



	\begin{bclogo}[logo=\bcattention, couleurBarre=red, noborder=true, 
		couleur=LightSalmon]{Important!}
		i dont think this simulation is well designed to simulate a real cyberattack as they represent those attacks by just changing some configuration values in the simulation to demonstrate the potential effects of an attack therfore i dont think implemeting a real life example of an IDS is possible here
	\end{bclogo}
	




	\section{conclusion}
	in \path{Securing the Smart Grid A Comprehensive Compilation of Intrusion Detection and Prevention Systems (PANAGIOTIS I. RADOGLOU-GRAMMATIKIS etc.).pdf}

	some of the methodes used to test SG security were 
	\begin{itemize}
		\item testing on a real life testbed and used real cyberattacks targeted at them to test IDS
		\item some used snort IDS on real hardware
	\end{itemize}

	not much sources or data was provided on those tests or how to replicate them
	


	i searched for ways to simulate cyberattacks on a smart grid but i found nothing

	i searched for other simulation but they are mostly either not implemented or incomplete


	i think the remaining solution would be to either implement an IDS for gridlab or a WIPS but i have no hardware to test on or any simulation which leave the only option to try to implement it on my home router as if it is a smart grid component

	would like to hear your opinion soon. thank you.



	%\chapter*{Acknowledgments}
\addcontentsline{toc}{chapter}{Acknowledgments}


%(This section allows you to thank all the people who have participated in the successful development of the end of studies project, and especially when writing your thesis. This \textbf{must not exceed 1~page maximum}.)



We thank ALLAH the Almighty, Great and Merciful for us having given confidence, good health, patience, will and courage to complete this work.

We thank Mr"Salim Benayoune" for his support and commendable effort, for his precious guidance, and assistance during the completion of this work.

We thank our parents who have surrounded us with strength and devoted love ever since we were born.
	%\chapter*{Dedication}
\addcontentsline{toc}{chapter}{Dedication}

(In this section, you dedicate this thesis to important people for you. This \textbf{should not also exceed 1~page}.)  % Optional
	
	%\addcontentsline{toc}{chapter}{Abstracts}
	%\begin{abstractAr}





		تتعرض موثوقية الشبكة الذكية وأمنها لخطر كبير بسبب الاعتماد المتزايد على تقنيات الشبكة الذكية، والتي جلبت نقاط ضعف جديدة تتمثل في التهديدات السيبرانية. العديد من أنظمة كشف التسلل التقليدية \LR{(IDSs)} غير قادرة على اكتشاف معظم هذه التهديدات بسبب التعقيد وبيانات الشبكة الذكية. تقدم هذه الأطروحة نظام كشف التسلل إلى الشبكة القائم على التعلم العميق \LR{(DL-NIDS)} الذي يعالج هذه المشكلة
		
		  تستخدم \LR{NIDS} المستندة إلى \LR{DL} بنية مبتكرة تستخدم الشبكات العصبية التلافيفية \LR{(CNNs)} والشبكات العصبية المتكررة \LR{(RNNs)} لاكتشاف الانحرافات في حركة مرور شبكة الشبكة الذكية. وهذا يشمل الأنماط العادية وكذلك تلك المؤذية.
		
		  ولتقدير أدائها، تم تدريب هذه النماذج على بيانات حركة مرور الشبكة الحقيقية ثم تم اختبارها على مجموعة جديدة من البيانات. وعلى النقيض من أجهزة كشف الهوية التقليدية، هناك تحسن كبير في كل من المتانة ودقة الكشف كما هو موضح في النتائج. كما ثبت أن النظام قادر على اكتشاف مجموعة متنوعة من هجمات DoS وDDoS. ولذلك، فإن هذا البحث يعزز موثوقية وسلامة أنظمة البنية التحتية الحيوية هذه من خلال المساهمة في إيجاد حلول أفضل للأمن السيبراني للشبكات الذكية.
		
		

\end{abstractAr}

\begin{keywordsAr}
	(الشبكة الذكية، الأمن السيبراني، التعلم العميق، هجمات حجب الخدمة، الشبكة العصبية التلافيفية، الذاكرة الطويلة قصيرة المدى)
\end{keywordsAr}

\newpage 
	%\begin{abstractEn}
	

	The smart grid's reliability and security are at significant risk due to the increasing reliance on smart grid technologies, which have brought in new vulnerabilities which is cyber threats. Many traditional intrusion detection systems (IDSs) are unable to detect most of these threats because of the complexity and smart grid data.This thesis introduces a deep learning-based network intrusion detection system (DL-NIDS) that tackles this issue

	The DL-based NIDS uses an innovative architecture utilizing convolutional neural networks (CNNs) and recurrent neural networks (RNNs) to discover deviations in smart grid network traffic. This includes normal patterns as well as those that are maleficent.

	To estimate its performance, these models were trained on real life network traffic data then tested on a new set of data. In contrast to conventional IDSs, there is a dramatic improvement in both robustness and detection accuracy as shown by results. The system is also demonstrated to be capable of detecting a variety of DoS and DDoS attacks. Therefore, this research enhances the dependability and safety of such critical infrastructure systems by contributing towards better cybersecurity solutions for smart grids.


\end{abstractEn}

\begin{keywordsEn}
	(smart grid, cybersecurity, deep learning, denial of service attacks, Convolutional neural network , Long short-term memory)
\end{keywordsEn}



\newpage

	%%\begin{abstractFr}
%	
%	(Vous insérez ici le résumé du manuscrit en français, dont le nombre de mots \textbf{ne doit pas dépasser 200}.)
%	
%\end{abstractFr}
%
%\begin{keywordsFr}
%	(6 mots clés au maximum doivent être séparés par des virgules ",")
%\end{keywordsFr}




\begin{abstractFr}


La fiabilité et la sécurité du réseau intelligent sont considérablement menacées en raison du recours croissant aux technologies de réseau intelligent, qui ont introduit de nouvelles vulnérabilités que sont les cybermenaces. De nombreux systèmes de détection d'intrusion (IDS) traditionnels sont incapables de détecter la plupart de ces menaces en raison de la complexité et des données des réseaux intelligents. Cette thèse présente un système de détection d'intrusion réseau basé sur l'apprentissage profond (DL-NIDS) qui s'attaque à ce problème.

  Le NIDS basé sur DL utilise une architecture innovante utilisant des réseaux neuronaux convolutifs (CNN) et des réseaux neuronaux récurrents (RNN) pour découvrir les écarts dans le trafic du réseau intelligent. Cela inclut les schémas normaux ainsi que ceux qui sont maléfiques.

  Pour estimer ses performances, ces modèles ont été formés sur des données réelles de trafic réseau puis testés sur un nouvel ensemble de données. Contrairement aux IDS conventionnels, les résultats montrent une amélioration considérable de la robustesse et de la précision de la détection. Le système s’est également révélé capable de détecter diverses attaques DoS et DDoS. Par conséquent, cette recherche améliore la fiabilité et la sécurité de ces systèmes d’infrastructures critiques en contribuant à de meilleures solutions de cybersécurité pour les réseaux intelligents.


  \end{abstractFr}



  \begin{keywordsEn}
	(grille intelligente, cybersécurité, apprentissage profond, attaques par déni de service, réseau neuronal convolutif, mémoire à long terme)
\end{keywordsEn}


	%\clearpage
	
	%\tableofcontents
	%\listoffigures
	%\listoftables
	%\listofalgorithms  % Optional

\mainmatter

\backmatter


\end{document}