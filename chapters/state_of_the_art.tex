\chapter{State of the Art} \label{chap:state_of_the_art}


\section*{Introduction}
Electricity grids, often called “grids”, are an indispensable part of the modern energy infrastructure. They have the capabilities of generating, transmitting, distributing and controlling power. Grids have evolved from local small scale grids to interconnected systems that respond to growing demand and changes in technology. Their contribution to economic growth cannot be overemphasized.

Moreover, amid changing dynamics in the energy sector, the advent of “smart grid” could also change the landscape. That is because smart grids employ data, automation and connectivity towards improving on energy management efficiency and ensuring sustainability. This chapter presents how grid systems have evolved over time, it gives a description of some challenges as well as opportunities related to smart grid technology while shaping the future of energy.



\newpage

\section{Definition smart grid}
The Smart Grid is a comprehensive electrical network that employs cutting-edge communication technologies, computational intelligence, and cybersecurity protocols throughout the entire process of generating, transmitting, distributing, and consuming electricity. Its objective is to establish a system that is environmentally friendly, secure, dependable, adaptable, energy-efficient, and environmentally sustainable. While the ultimate vision of the Smart Grid is ambitious, its practical implementation demands careful evaluation of costs, rigorous testing, and validation. Introducing new functionalities can occur autonomously, with each necessitating justification and a reasonable return on investment. The compatibility of open systems facilitates smooth integration into the Smart Grid once the technologies have been validated \cite{gharavi2011smart}.
 The National Institute of Standards and Technology (NIST), operating within the U.S. Department of Commerce, has classified the smart grid into seven distinct domains, as illustrated in Figure \ref{fig:NIST}. A concise overview of these domains and their stakeholders is provided in Table \ref{table:domains}.\cite{gopstein2021nist}
\begin{figure}[h]
	\centering
	\includegraphics[width=\textwidth]{figures/nist.PNG}
	\caption{The NIST Conceptual Model for SG \cite{gopstein2021nist}}
	\label{fig:NIST}
\end{figure}
\begin{table}[h]
    \centering

    
% \begin{tabularx}{\textwidth}{|X|X|}
% \hline
% \textbf{Domain} & \textbf{Roles/Services in the Domain} \\
% \hline
% 1 Customer & The end users of electricity. May also generate, store, and manage the use of energy. Traditionally, three customer types are discussed, each with its own sub-domain: residential, commercial, and industrial. \\
% \hline
% 2 Markets & The facilitators and participants in electricity markets and other economic mechanisms used to drive action and optimize system outcomes. \\
% \hline
% 3 Service Provider & The organizations providing services to electrical customers and to utilities. \\
% \hline
% 4 Operations & The managers of the movement of electricity. \\
% \hline
% 5 Generation Including DER & The producers of electricity. May also store energy for later distribution. This domain includes traditional generation sources and distributed energy resources (DER). \\
% \hline
% 6 Transmission & The carriers of high voltage electricity over long distances. May also store and generate electricity. \\
% \hline
% 7 Distribution & The distributors of electricity to and from customers. May also store and generate electricity. \\
% \hline
% \end{tabularx}
\begin{tabular}{|p{0.15\textwidth}|p{0.8\textwidth}|}
    \hline
    \textbf{Domain} & \textbf{Roles/Services in the Domain} \\
    \hline
    1 Customer & The end users of electricity. May also generate, store, and manage the use of energy. Traditionally, three customer types are discussed, each with its own sub-domain: residential, commercial, and industrial. \\
    \hline
    2 Markets & The facilitators and participants in electricity markets and other economic mechanisms used to drive action and optimize system outcomes. \\
    \hline
    3 Service Provider & The organizations providing services to electrical customers and to utilities. \\
    \hline
    4 Operations & The managers of the movement of electricity. \\
    \hline
    5 Generation Including DER & The producers of electricity. May also store energy for later distribution. This domain includes traditional generation sources and distributed energy resources (DER). \\
    \hline
    6 Transmission & The carriers of high voltage electricity over long distances. May also store and generate electricity. \\
    \hline
    7 Distribution & The distributors of electricity to and from customers. May also store and generate electricity. \\
    \hline
    \end{tabular}
    \caption{Domains and their associated roles/services \cite{gopstein2021nist}}
    \label{table:domains}
\end{table}



\section{Smart grid attributes}
Many smart grid advocates cite some or all of its following attributes as representative of its promise:
\firmlist
\begin{itemize}


\item \textbf{  Efficiency:} Capable of meeting growing consumer demand without the need for additional infrastructure.

\item \textbf{  Flexibility: }Able to accept energy from various sources, including solar and wind, with the same ease as traditional fuels like coal and natural gas. It can integrate new technologies, such as energy storage, as they become commercially viable.

\item \textbf{  Empowering: }Facilitating real-time communication between consumers and utility providers, allowing consumers to adjust their energy usage based on factors like price and environmental concerns.

\item \textbf{  Opportunistic:} Creating new markets and opportunities by leveraging plug-and-play innovations whenever suitable.

\item\textbf{   Focus on Quality:} Able to deliver reliable power without disruptions, ensuring the smooth operation of digital technologies crucial to our economy.

\item \textbf{  Resilience:} Increasingly resistant to cyber attacks and natural disasters through decentralization and the implementation of smart grid security measures.

\item \textbf{  Environmental Sustainability:} Contributing to the mitigation of climate change and offering a viable path towards reducing the environmental impact of electricity generation. \cite{el2014smart}
\end{itemize}


\section{Differences between Traditional grid and Smart grid }
Table \ref{tab:comparison} offers a thorough comparison of the conventional power grid with the smart grid. In contrast to the traditional grid where customers play a passive role, the smart grid actively engages them through bi-directional communication technologies. For instance, rooftop photovoltaic solar panels produce electricity during the day, enabling customers to sell surplus energy back to the grid. At night, these panels continue to power home appliances as usual. Moreover, the smart grid incorporates innovative technologies like distributed generation, electric vehicle charging and discharging, and Flexible Alternating Current Transmission Systems (FACTS) to improve energy distribution and management.\cite{zhang2014smart}
\begin{table}[h]
  
	\caption{Comparison between conventional grid and smart grid \cite{miller2008understanding}}
    

	\begin{tabular}{|p{3cm}|p{6cm}|p{6cm}|}
	\hline
	Aspects & Conventional Grid & Smart Grid \\
	\hline
	Interaction between Grid and Customers & Customers passively accept service from grid & Customers participation on the grid action \\
	\hline
	Renewable Energy Integration & Having trouble with renewable penetration & Integration with renewable resources enhancement \\
	\hline
	Options for Customers & No choice for customer, monopoly market & With digital market trading, PHEV, introduce bids and competition, more choice for customer \\
	\hline
	Options on Power Quality (PQ) & No choice on power quality, no price plan options for consumers & Power quality levels for different consumers \\
	\hline
	System Operation & Ageing power assets, no efficient operation & Assets operating optimization, less power loss \\
	\hline
	Protection & Only rely on protection devices, fault detect manually & Have capability of self-healing, less damage affected by fault \\
	\hline
	Reliability and Security & Susceptible to physical and cyber attack & More reliable for national security and human safety \\
	\hline
	\end{tabular}

	\label{tab:comparison}
\end{table}

\section{Major systems}


\subsection{Smart infrastructure system}
The smart infrastructure system consists of three main components: the smart energy subsystem, the smart information subsystem, and the smart communication subsystem. Within the smart energy subsystem, activities such as electricity generation, transmission, distribution, and consumption are integrated. The smart information subsystem includes functions like smart metering and advanced monitoring and management of the smart grid network. The smart communication subsystem facilitates wired and wireless communication between networks, devices, and applications to establish connectivity throughout the network \cite{shafiullah2013smart}.
\subsection{Smart management system}
The smart grid's intelligent management system offers advanced services in monitoring and control. As innovative management, monitoring, and control applications evolve, smart grid technology becomes more sophisticated, contributing actively to the advancement of a sustainable power system. Within the smart management system are functions such as enhancing energy efficiency, balancing supply and demand, controlling emissions, reducing operational costs, and maximizing utility. This system utilizes modern machine learning and optimization tools to create a resilient and efficient smart management framework \cite{shafiullah2013smart}.
\subsection{Smart protection systems}
The smart protection system within the smart grid offers services related to reliability, safeguarding against failures, and ensuring security and privacy. By incorporating advanced protection devices and monitoring tools, the system enhances the reliability, security, and privacy of the network. Alongside smart infrastructure planning, efficient management, and intelligent protection systems play a role in managing operations effectively, protecting against failures, and addressing cybersecurity and privacy concerns within the network. Figure \ref{fig:Classification} illustrates a typical technological framework of the smart grid \cite{shafiullah2013smart}.
\begin{figure}[h]
	\centering
	\includegraphics[width=12cm]{figures/Classification.PNG}
	\caption{Classification of the Smart Infrastructure System, the Smart Management System, and the Smart Protection
	System \cite{fang2011smart}}
	\label{fig:Classification}
\end{figure}

\newpage







\section{Smart Grid Technologies }
A smart grid employs a diverse array of technologies and communication networks to enhance the management of power generation, transmission, and distribution. It also provides customers with the ability to have real-time control over their energy consumption \cite{WhatIsSmartgrid}.
\subsection{Major Smart Grid Technologies}
\subsubsection{Advanced Demand Forecasting}
Utilizing data analytics and machine learning (ML), advanced demand forecasting techniques produce forecasting reports through autoregressive integrated moving average (ARIMA) and various statistical methods.

A crucial aspect of smart grid management, ARIMA forecasting predicts both annual electricity consumption and hourly electricity prices.

Furthermore, ARIMA forecasting serves as an extra layer of verification, aiding in the identification of cyber intrusion attempts on smart meters used to measure electricity usage for residential and commercial consumers \cite{WhatIsSmartgrid}.
\subsubsection{Advanced Metering Infrastructure (AMI)}
Advanced Metering Infrastructure (AMI) is a unified system comprising communication networks, data management systems, and intelligent meters designed to enhance customer service, energy efficiency, and cost management.

AMI facilitates two-way communication between customers and utilities, offering a wide array of advantages to the smart grid. These include forecasting consumption, improving revenue collection and theft detection, detecting faults and outages, measuring losses, and implementing time-based pricing \cite{WhatIsSmartgrid}.  
\subsubsection{Big Data}
Smart grid data possesses three fundamental characteristics: high velocity, extensive volume, and diverse variety. Managing this large volume of data in a timely manner with limited resources poses a significant challenge for smart grids. This is where big data analytics becomes pivotal, offering the potential to boost asset utilization, efficiency, system reliability, and customer satisfaction.

Without big data analytics in the smart grid, the assessment of petabytes of data generated by smart grid devices would be impractical. Big data captures and analyzes unstructured data from various endpoints within a smart grid.

Moreover, big data facilitates efficient cost reduction, optimal resource distribution, and improved customer service \cite{WhatIsSmartgrid}.
\subsubsection{Distributed Energy Resources (DERs)}
Distributed Energy Resources (DERs) supply energy and improve local reliability, enhancing grid stability and optimizing on-site fuel utilization.

DERs encompass various technologies such as electric vehicles, solar panels, small natural gas generators, and controllable loads like electric water heaters and HVAC systems.

Efficient integration of DERs enhances grid service quality and reliability. For instance, photovoltaic systems (PVs) utilize the photovoltaic effect to convert sunlight into electricity, which is then transformed into alternating current by an inverter. The primary advantage of PV systems is reduced utility bills due to decreased reliance on grid-provided electricity \cite{WhatIsSmartgrid}.
\subsubsection{Non-intrusive Load Monitoring (NILM)}
Non-intrusive load monitoring (NILM), also known as non-intrusive appliance load monitoring (NIALM), discerns the specific energy consumption of households and industrial sites.

By disaggregating the total energy usage (from active appliances) into individual components and offering diagnostic insights, NILM aids in identifying energy-intensive or faulty appliances.

Moreover, consumers can optimize the timing of usage for energy-intensive appliances to minimize costs, and monitor and control energy expenses based on their power consumption \cite{WhatIsSmartgrid}. 
\subsubsection{Vehicle-to-Grid (V2G)}
Also known as vehicle-grid integration (VGI), vehicle-to-grid (V2G) technology transfers unused power from a vehicle into the smart grid. An electric vehicle (EV) battery is a cost-efficient form of energy storage.

 

V2G helps balance electricity consumption spikes and reduce overload on the power grid during peak hours.

 

For example, V2G can feed energy (unused battery capacity) back to the power grid from an electric car’s battery to improve grid stability and maximize the benefits of renewable energy \cite{WhatIsSmartgrid}.

 \subsection{Established and Emerging Smart Grid Communication Networks}
 \subsubsection{HAN}
 A smart meter supplies power to household appliances via the Home Area Network (HAN), which utilizes different technologies such as Bluetooth, Wireless Ethernet, Wired Ethernet, and Zigbee. The HAN links home appliances with the smart meter, which detects power usage and transmits this information to the server for billing purposes \cite{WhatIsSmartgrid}.
 \subsubsection{NAN}
 A Neighborhood Area Network (NAN) is an external access network that links distribution automation devices and smart meters to WAN gateways such as RF (radio frequency) collectors and field devices (like Intelligent Electronic Devices (IEDs)). NAN allows for customer data collection and facilitates communication within the WAN-premise area \cite{WhatIsSmartgrid}.
 \subsubsection{WAN}
 A wide area network (WAN) uses fiber optics, 3G/LTE (Long Term Evolution)/GSM (Global System for Mobile Communication), or WiMAX (Worldwide Interoperability for Microwave Access) for communication between a smart meter, suppliers, and the utility server. A smart meter sends notifications it receives (via HAN) from the devices to the suppliers using WAN \cite{WhatIsSmartgrid}. 
 \subsubsection{LoRaWAN}
 LoRa (Long Range) is a popular IoT (Internet of Things) technology known for its long-range capability and low-power wireless platform, making it well-suited for various applications including energy management, infrastructure efficiency, and disaster prevention.

 Implementing smart electricity metering solutions and smart grid networks using the LoRaWAN® (Long Range Wide Area Network) protocol allows for improved understanding of power demand, efficient detection of power outages, enhanced connectivity, and identification of underperforming assets.
 
 Additionally, LoRaWAN is globally compatible and ensures seamless transmission without interference for remote reading of heat meter consumption data \cite{WhatIsSmartgrid}.
\section{Components of the Smart Grid }  
There are many components, but will talk about the most important ones.
\subsection{Smart Meters}
The interplay between smart meters and smart grids is depicted in Figure \ref{fig:SmartMeter}. From the perspective of the smart grid, the applications primarily revolve around leveraging smart meters to facilitate the coordination of various electrical devices, thereby achieving a dependable power system. Simultaneously, these applications strive to enhance the performance and efficiency of smart metering. These objectives align with the defining characteristics of smart grids, which drive the advancement of smart meter technologies.\cite{chen2023control}
\begin{figure}[h]
	\centering
	\includegraphics[width=\textwidth]{figures/SmartMeter.PNG}
	\caption{Applications from smart grid and smart meter perspectives. \cite{chen2023control}}
	\label{fig:SmartMeter}
\end{figure}
%new add
%\newpage
\subsection{Advanced Distribution Management Systems }
An ADMS is a software platform designed to support the comprehensive suite of tasks related to managing and optimizing the distribution of electricity. It encompasses functions that automate outage recovery and enhance the effectiveness of the distribution grid. These functions being developed for electric utilities include fault location, isolation, and restoration; optimization of voltage and reactive power; energy conservation through voltage reduction; management of peak demand; as well as support for microgrids and electric vehicles \cite{avazov2016advanced}.

% \firmlist
% \begin{itemize}


% \item \textbf{ Optimizing Power Flow:} ADMS dynamically adjusts power distribution to ensure efficient energy transfer across the grid.

% \item \textbf{  Real-Time Monitoring: }They keep a vigilant eye on the grid, promptly detecting issues like outages.

% \item \textbf{  Rerouting Electricity: }When problems arise, ADMS reroutes electricity to minimize disruptions.

% \end{itemize}


\subsection{Super conducting cables}
These components are utilized for transmitting electricity over extended distances and employing automated monitoring and analysis tools. These tools have the capability to identify faults independently or predict potential cable failures by analyzing real-time data, weather conditions, and the history of outages \cite{elprocus_smartgrid}.
\subsection{Circuit breakers}
The circuit breaker, a component responsible for protecting the power system from the damage that can be caused by spikes in electric current, a circuit breaker will shut down the entire power system to avoid causing damage by the excessive current. The smart and improved version of it is the smart circuit breaker, which has wireless connection capabilities to monitor the systems behaviour \cite{circuitbreaker}.
\subsection{Collector nodes}

Collector nodes are pivotal in the Smart Grid, serving as points of data collection and distribution between energy suppliers and customers. They enable a two-way communication network within the grid, relaying information from customer premises to the utility control center and transmission/distribution substations. Collector nodes facilitate efficient monitoring and management of energy usage \cite{cunjiang2012architecture}.

\section{Challenges and Considerations  }  
\subsection{Stakeholder Engagement}
At the early stages of smart grid implementations, stakeholders’ negative perceptions can derail even the most beneficial project, especially when the proponents fail to pay close attention to the educational aspects. Advocates need to be able explain and clearly identify the benefits of each component of the smart grid to the customers that are the potential key to service success \cite{el2014smart}.
\subsection{Fear of obsolesce}
As many technology users (computers, smart phones, etc.) are painfully aware, the adoption of new tools can open the door to new and additional costs that may only be borne by the eventual consumer. This fear can be addressed through the development of interoperability standards and backward compatibility of technologies \cite{el2014smart}.
\subsection{Cybersecurity}
Without a shred of doubt, cybersecurity stands out as one of the foremost and intricate challenges confronting IoT devices. Sensors, devices, and networks connected to the internet are persistent targets for various online threats like probing, espionage, ransomware, theft, and potential destruction. Considering that an IoT-driven smart grid can encompass potentially millions of interconnected nodes spread across extensive geographic regions, it emerges as the most susceptible to substantial cyber assaults. Consequently, a cyber-attack on such a system would have devastating consequences, leading to significant financial losses and potentially bringing entire countries to a standstill. The diagram in Figure \ref{fig:attacks} illustrates the number of articles reviewed per year of publication and smart grids impacted by cyber-attacks. Hence, security stands as a major hurdle in both the deployment and operation of IoT-based smart grid networks.\cite{kimani2019cyber}.
\begin{figure}[h]
	\centering
	\includegraphics[width=12cm]{figures/attack.png}
	\caption{Estimate: cyber attacks will increase exponentially \cite{mdpi-link}}
	\label{fig:attacks}
\end{figure}

\subsection{Data privacy}
Privacy is a critical concern within smart grid networks, prompting significant questions about the creation of policies regarding user data privacy. These questions revolve around several key points: Who owns the customer data? How is access to and usage of customer data regulated? What measures exist to protect the privacy and security of customer data from potential risks like surveillance or illicit activities? Is it permissible to sell or transfer customer data, and under what circumstances and for whose benefit? In areas with retail choice, are measures necessary to ensure that competing electricity providers have equal access to customer data compared to the incumbent utility?

In competitive environments among electricity providers, access to users' electricity usage patterns and behavioral information holds significant importance. Providers or their representatives may use this data to develop business strategies and create tailored packages or offers. In an open market scenario, some data may be disclosed after offers are made public, providing a level playing field for information access. However, if privacy is compromised beforehand, with specific user data available to only certain parties, these electricity providers could potentially gain unfair advantages. Therefore, effective privacy policies are essential to prevent the exploitation of unfair means in shaping business strategies.

The integration of Information and Communication Technologies (ICTs) into smart grid operations introduces various privacy concerns. Depending on how a consumer uses and recharges electricity \cite{zeadally2013towards}.

\subsection{Cost of Implementation:}

Estimating Smart Grid costs poses challenges due to several factors. Integrating digital technology into Smart Grids introduces complexities, as the failure rates and life expectancy of embedded assets differ from traditional grid technologies. For instance, a substation transformer designed for 40 years may be coupled with information technology lasting 10, 15, or 20 years, necessitating careful cost considerations for upgrades. Additionally, the rapid obsolescence of digital tech complicates estimates, as advancing communications and computational capabilities may render Smart Grid components obsolete before their intended lifespan ends. 

Moreover, the evolution of Smart Grid technologies is expected to outpace conventional tech in terms of cost reduction and advancements. However, uncertainties persist, particularly with new and unproven Smart Grid technologies. If their performance is subpar or degrades unexpectedly, it could jeopardize the entire technology's viability and business plan. As Smart Grid component costs decrease rapidly due to maturation and increased production, estimating replacement costs becomes challenging\cite{smartgrid}.
\begin{figure}[h]
	\centering
	\includegraphics[width=12cm]{figures/cost.PNG}
	\caption{Grid Component Costs \cite{smartgrid}}
	\label{fig:costs}
\end{figure}
\subsection{Regulatory Frameworks}
Electric distribution systems across Europe are encountering significant hurdles stemming from climate change objectives, evolving market frameworks, and technological advancements. These factors will have a profound impact on the responsibilities of distribution system operators. The challenges' nature and magnitude are primarily influenced by Europe's vision and strategies regarding climate and energy. This research aims to identify which policies might pose obstacles to innovation in distribution grids and the adoption of advanced smart grid solutions developed within the UNITED-GRID project. Following an in-depth examination of emerging policy priorities within the energy and climate framework, as well as electricity market design, and subsequent consultations with three partner distribution system operators, five key barriers have been pinpointed. The findings indicate that ambitious decarbonisation targets and shifting expectations regarding the role of distribution system operators in the energy landscape necessitate more adaptable and efficient network management. However, rigid income frameworks, insufficient incentives for innovation, and regulatory uncertainties impede the modernization of distribution systems. It can be inferred that these concerns heighten the risks for distribution system operators and must be taken into account by research initiatives and developers of smart grid solutions to successfully implement and achieve market adoption of the developed solutions \cite{rossi2020study}. 



\section{Related works}
Due to the critical importance of the smooth operation of the smart grid, detecting malicious behaviour towards it is of utmost mportance. Analysing network traffic going in or going out of the smart grid infrastructure is required for detecting malicious activity, as intrusions usually have patterns and signatures that are detectable but not always. That's why many studies have turned to the emerging artificial intelligence technologies that can detect patterns humans can't detect based on statistical probabilities, which are obtained by analysing previous similar data.

For example, in \cite{related1}, the author proposes an AI-based security solution for the data-driven parts of the smart grid that uses multiple classifiers to compare their performance, which are K-Nearest Neighbour, Neural Network, Decision Tree, and Random forest. He also used the PSO search algorithm for selecting features from a given subset of features. The proposed model was trained on the KDD99 and NSLKDD datasets. It is intended to be a binary classification model of network traffic, with the result of prediction being either normal or anomaly, but it also has multiclass classification that can predict attack categories like DoS, R2L, U2R, and Probe.



The proposed model has of six phases:
\firmlist
\begin{enumerate}
	\item Data reading
	\item Data preprocessing
	\item Passing optimal features to machine learning selected models
	\item Training the model with 70\% of the dataset, then testing it with the 30\% of the dataset
	\item Experiment phase
	\item Evaluation 
\end{enumerate}

The author then displayed an extensive evaluation of the model with the previously mentioned evaluation criteria and tested the proposed model on various known reconfiguration tools like nmap, portweep, and more. The results of each classifier were closely matched, with random forest having the highest evaluation scores by a tiny margin on the KDD99 dataset with a precision of 99.8\% for attacks and 98.5\% for normal traffic, a recall of 99.6\% for attacks and 99.3\% for normal, and a f1 score of 99.7\% for attacks and 98.9\% for normal. As for the NSLKDD dataset, the performance was closely matched to that of the KDD99 dataset, with a few slight differences in each metric.






In order to increase detection accuracy, the research paper \cite{related2} suggests a parallel structure utilising Recurrent Neural Network (RNN) classifier models, namely Long Short-Term Memory (LSTM) and Gated Recurrent Units (GRU). A dataset derived from an experimentally built SDN-based SCADA topology is used to train and evaluate the model. To improve the model's performance even more, transfer learning is used, and a further five percent improvement was attained. The results show that DDoS assaults in SDN-based SCADA systems can be successfully detected by the suggested RNN deep-learning classifier model.


Another example is this paper \cite{related3}, in which the author was able to enhance SCADA systems security against DDoS attacks using machine learning techniques. The machine learning that were used were J48, Naive Bayes, and Random Forest. For training and evaluating the algorithms, the authors used the KDD99 dataset and also employed some pre-processing methods. The data indicate that Random Forest had an accuracy rate of 99.99\% while Naïve Bayes came in second with an accuracy of 97.74\%. Consequently, there exists a good basis for improving critical infrastructure security by knowing how machine learning algorithms detect attack patterns in SCADA systems.





In \cite{related4}, the author tested the effectiveness of Snort and Suricata, two open-source intrusion detection systems (IDSs), for precisely identifying hostile traffic on networks. A hybrid form of SVM and fuzzy logic was also used in the study, and it resulted in increased detection accuracy. However, using an optimised SVM with the firefly method produced better results, with a false-negative rate (FNR) of 2.2\% and a false-positive rate (FPR) of 8.6\%. This result suggests a noteworthy enhancement in performance. The comparison of the two IDSs at a high network speed of 10 Gbps and the use of hybrid and optimised machine learning methods to improve Snort's functioning are what make this work novel.









\newpage


\section{Conclusion}
The smart grid revolution is a journey that has no end point, but an ongoing need for greater efficiency, reliability and sustainability. It is worth noting that as much as there are challenges associated with adoption of this technology, the benefits likely to be realized by a smart grid surpass its drawbacks. This paper identifies prospects where innovation, collaboration and consumerization can enhance the capabilities of this unraveled technology.

A more intelligent network opens up possibilities for a future which sees clean energy sources like solar and wind power being integrated seamlessly into homes and businesses taking an active role in management of power and virtually live without power failures. It is a world that guarantees our children have access to a safe and sustainable energy infrastructure.

With smart grids, le's take it further to make the future brighter for all.