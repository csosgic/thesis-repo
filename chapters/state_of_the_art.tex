\chapter{State of the Art} \label{chap:state_of_the_art}


\section*{Introduction}
Electricity grids, commonly referred to as “grids,” play a vital role in modern energy infrastructure. They facilitate power generation, transmission, distribution, and control. Over time, grids have evolved from localized systems to interconnected networks, adapting to meet increasing demands and technological advancements. These grids contribute significantly to economic and societal progress.

Amidst dynamic changes in the energy landscape, the emergence of the “smart grid” presents transformative possibilities. Leveraging data, automation, and connectivity, smart grids enhance energy management and promote sustainability. In this chapter, we delve into the evolution of grid systems and explore the challenges and opportunities associated with smart grid technology, shaping the future of energy.
\newpage


\section{smart grid}
\subsection{definition}
The term "smart grid" describes a modern infrastructure concept developed collaboratively within the power industry. It integrates information and communication technologies into electricity generation, delivery, and consumption processes to enhance efficiency, reliability, and sustainability. Various names, such as intelligent grid or grid wise, have been used interchangeably. The Electric Power Research Institute (EPRI) defines the smart grid as incorporating technology into every aspect of the electricity system to minimize environmental impact, improve reliability, and reduce costs. The International Electrotechnical Commission (IEC) emphasizes modernizing the electric grid by integrating electrical and information technologies. The Canadian Electricity Association highlights communication, integration, and automation as key themes within smart grid systems. Despite early skepticism, the smart grid represents a transformative advancement comparable to electrification in the 20th century, offering significant opportunities for modernizing energy systems.\cite{bar73}
\subsection{Why Smart Grid?}
The adoption of the Smart Grid has become crucial due to several factors. Firstly, since approximately 2005, there has been a growing interest in the Smart Grid driven by the recognition that Information and Communication Technology (ICT) offers significant opportunities for modernizing electrical network operations. Secondly, the urgent need to de-carbonize the power sector emphasizes the importance of effective monitoring and control, which the Smart Grid enables. Additionally, various specific reasons have converged to generate interest in the Smart Grid, emphasizing its necessity in today’s energy landscape.\cite{ekanayake2012smart}
\subsection{smart grid vs grid}
gggg \cite{farhangi2009path}
\section{Related Works}
\section{Synthesis and Discussion}
\section*{Conclusion}
