%\chapter{Contributions} \label{chap:contributions}
%
%(This part includes all the contributions proposed in your project. You describe the adopted approach and methodology and you explain how you carried out your project. The results obtained are also presented, analyzed and discussed. This part may consist of one (01) or two (02) chapters maximum, and \textbf{should not exceed 20~pages}. The general structure is as follows:)
%
%\section*{Introduction}
%
%\section{Theoretical Proposal}
%(This section may include the following: Project description, formal or semi-formal project design, system architecture, process used in project development, etc.)
%
%\section{Implementation et Experiments}
%\section*{Conclusion}










\section{Introduction}
An intrusion detection system is a piece of hardware or software that is responsible for detecting suspicious and malicious activity, and in a network or an information system, the anomaly can either be reported to a systems administrator or saved to a security information and even management system (SIEM), the SIEM combines the output from multiple sources, then uses some filtering techniques to decide if the reported activity is malicious. \cite{1}Intrusion detection systems are categorized into 2 categories based on the location of the detection, which are either network or host-based (HIDS or NIDS), There are also two primary methods of intrusion detection: signature-based and anomaly-based. \cite{2}
