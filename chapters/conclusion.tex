\chapter*{General Conclusion} \label{chap:conclusion}
\addcontentsline{toc}{chapter}{General Conclusion}





The proposed deep learning-based network intrusion detection system effectively addresses the issue of protecting smart grid infrastructure from distributed denial-of-service (DDoS) and denial-of-service (DoS) attacks. The system leverages convolutional neural networks (CNN) and long-short-term memory (LSTM) algorithms to identify and prevent malicious network traffic. This solution is particularly relevant in the context of smart grids, where the reliability and security of communication networks are crucial for efficient and safe operation.



\section*{contribution}



\begin{itemize}
    \item Developing a dependable and reliable intrusion detection system that can detect DoS and DDoS attacks in the smart grid.
    \item Improved accuracy and efficiency as an AI model is capable of detecting patterns that are undetectable to a human.
    \item Scalability and adaptability because it is important to meet the constantly increasing demand in the smart grid.
\end{itemize}


\section*{Perspectives}


Although our solution based on real-life network traffic data for detecting DoS and DDoS attacks has shown promising results with good accuracy, to further enhance the accuracy and effectiveness of malicious activity detection, there is a need to explore additional real network traffic data in a variety of situations and use cases and more network traffic pattern analysis. We propose:



\section*{Limitations}
\begin  {itemize}
    \item Better quality and availability of training data, because that can significantly impact the performance of the system. Future work should focus on developing methods to handle noisy or limited data.
    \item Real-time processing is required because The system's ability to process data in real-time is crucial for effective DDoS attack detection.
    \item Integration with existing systems, The system should be designed to seamlessly integrate with existing smart grid infrastructure and security systems to ensure a smooth transition and optimal performance, because if it doesn't, new hardware and systems will be required, which makes it a very expensive endeavour.
    \item Future research should explore the application of other deep learning architectures and techniques to further improve the system's performance and adaptability.
\end{itemize}




By addressing these limitations and perspectives, the proposed deep learning-based network intrusion detection system can be further refined and optimised to provide enhanced security and reliability for smart grid communication networks.