\chapter*{General Conclusion} \label{chap:conclusion}
\addcontentsline{toc}{chapter}{General Conclusion}





The proposed deep learning-based network intrusion detection system effectively addresses the issue of protecting smart grid infrastructure from Distributed Denial-of-Service (DDoS) attacks. The system leverages Convolutional Neural Networks (CNN) and Long Short-Term Memory (LSTM) algorithms to identify and prevent malicious traffic. This solution is particularly relevant in the context of smart grids, where the reliability and security of communication networks are crucial for efficient and safe operation.



\section*{contribution}


\begin{itemize}
    \item Development of a deep learning-based intrusion detection system: The system uses a combination of CNN and LSTM algorithms to identify DDoS attacks in smart grid communication networks.
    \item Improved accuracy and efficiency: The proposed system demonstrates enhanced accuracy and efficiency compared to traditional methods, making it a more effective solution for detecting and preventing DDoS attacks.
    \item Scalability and adaptability: The system is designed to be scalable and adaptable to various smart grid communication network configurations and traffic patterns.
\end{itemize}

    
    
\section*{Perspectives}

The applicability of this project is significant, as it can be integrated into existing smart grid infrastructure to enhance the security and reliability of communication networks. This can lead to improved overall performance and reduced downtime, ultimately benefiting both the grid operators and consumers. However, there are some limitations and perspectives to consider:


\section*{Limitations}
\begin{itemize}
    \item Data quality and availability: The quality and availability of training data can significantly impact the performance of the system. Future work should focus on developing methods to handle noisy or limited data.
    \item Real-time processing: The system's ability to process data in real-time is crucial for effective DDoS attack detection. Future improvements should focus on optimizing the system's processing speed and efficiency.
    \item Integration with existing systems: The system should be designed to seamlessly integrate with existing smart grid infrastructure and security systems to ensure a smooth transition and optimal performance.
    \item Future research directions: Future research should explore the application of other deep learning architectures and techniques to further improve the system's performance and adaptability.
\end{itemize}
    
    
    
    



By addressing these limitations and perspectives, the proposed deep learning-based network intrusion detection system can be further refined and optimized to provide enhanced security and reliability for smart grid communication networks.