\chapter*{General Introduction} 
\label{chap:introduction} 
\addcontentsline{toc}{chapter}{General Introduction}
%
%(The introduction, which \textbf{must not exceed 3~pages}, consists of the following four sections.)
%
%\section*{Project Background}
%(In this section, you describe the context in which your project is being processed.)
%
%\section*{Problem}
%(Here, you describe the problem that needs to be solved in the development of your thesis. It comes %directly from the theme proposed by your supervisor(s).)
%
%\section*{Proposed Solutions}
%(Here, you list the objectives of your thesis study, as well as the solutions you consider to %answer the addressed problem.)
%
%In this work, we propose...
%
%\section*{Document Plan}
%This thesis is organized as follows: In the first chapter, we...
%


\section*{Project Background}
The increasing reliance on smart grid technology has led to a significant expansion of the communication infrastructure, which has, in turn, created new vulnerabilities and opportunities for cyber threats. The smart grid communication infrastructure is a complex network of devices, systems, and protocols that facilitate the efficient and reliable transmission of electricity from generation to consumption. However, this increased connectivity and interdependence have created a heightened risk of cyber attacks, which can compromise the integrity and reliability of the grid.

% humanized 
Growing reliance on smart grid technology has necessitate the extensive augmentation of the communications infrastructure, which has inherently generated new points of vulnerability and exposure to cyber threats. A smart grid communication infrastructure is a sophisticated arrangement of devices, systems, and protocols that allows the secure and efficient delivery of electricity from generation to consumption. But as transmission interconnections expanded over time, so have the cyber threats that could — were an adversary so inclined — weaken the grid's credibility and robustness.



\section*{Problem}
The problem that needs to be solved in the development of this thesis is the lack of effective cybersecurity measures to protect the smart grid communication infrastructure from cyber threats. The increasing reliance on smart grid technology has created a heightened risk of cyber attacks, which can compromise the integrity and reliability of the grid. The potential consequences of a successful cyber attack on the smart grid communication infrastructure are severe, including power outages, data breaches, and even physical harm to people and property.

%human 
Problem To Be Addressed: The episodic development of this thesis are the need for effective cybersecurity mechanisms to secure the smart grid communication infrastructure from cyber threats. The greater use of smart grid technology has also increased the vulnerability to cyber attacks, whose consequences can be the disruption of the grid being attacked. The smart grid communication infrastructure is a prime target for cyber attacks, which can result in catastrophic outcomes such as blackouts, privacy breaches and even destruction of life and property.




\section*{Proposed Solutions}
This thesis aims to develop a deep learning-based intrusion detection system using LSTM to improve the accuracy and effectiveness of intrusion detection in smart grid communication infrastructure. The objectives of this study are:
 - To design and implement a deep learning-based intrusion detection system using LSTM for smart grid communication infrastructure.
 - To evaluate the performance of the proposed system using metrics such as accuracy, precision, recall, and F1-score.
 - To compare the performance of the proposed system with traditional rule-based IDS methods.


\section*{In this work, we propose...}
In this thesis, we propose a deep learning-based intrusion detection system using LSTM to improve the accuracy and effectiveness of intrusion detection in smart grid communication infrastructure. The proposed system will be designed and implemented using a combination of data preprocessing, feature engineering, and LSTM-based classification.


\section*{Document Plan}
This thesis is organized as follows: In the first chapter, we provide an overview of the smart grid communication infrastructure and the importance of cybersecurity. In the second chapter, we discuss the related works and the state of the art in intrusion detection systems. In the third chapter, we describe the proposed system and its components. In the fourth chapter, we present the experimental results and evaluation of the proposed system. Finally, in the fifth chapter, we conclude the thesis and discuss the implications of the results.