\chapter*{General Introduction} 
\label{chap:introduction} 
\addcontentsline{toc}{chapter}{General Introduction}
%
%(The introduction, which \textbf{must not exceed 3~pages}, consists of the following four sections.)
%
%\section*{Project Background}
%(In this section, you describe the context in which your project is being processed.)
%
%\section*{Problem}
%(Here, you describe the problem that needs to be solved in the development of your thesis. It comes %directly from the theme proposed by your supervisor(s).)
%
%\section*{Proposed Solutions}
%(Here, you list the objectives of your thesis study, as well as the solutions you consider to %answer the addressed problem.)
%
%In this work, we propose...
%
%\section*{Document Plan}
%This thesis is organized as follows: In the first chapter, we...
%


\section*{Project Background}
Growing reliance on smart grid technology has necessitate the extensive augmentation of the communications infrastructure, which has inherently generated new points of vulnerability and exposure to cyber threats. A smart grid communication infrastructure is a sophisticated arrangement of devices, systems, and protocols that allows the secure and efficient delivery of electricity from generation to consumption. But as transmission interconnections expanded over time, so have the cyber threats that could — were an adversary so inclined — weaken the grid's credibility and robustness.



\section*{Problem}
Problem To Be Addressed: The episodic development of this thesis is the need for effective cybersecurity mechanisms to secure the smart grid communication infrastructure from cyber threats. The greater use of smart grid technology has also increased the vulnerability to cyber attacks, whose consequences can be the disruption of the grid being attacked. The smart grid communication infrastructure is a prime target for cyber attacks, which can result in catastrophic outcomes such as blackouts, privacy breaches and even destruction of life and property.




\section*{Proposed Solutions}
This thesis is aimed at developing a deep learning-based intrusion detection system using LSTM and CNN with the objective of enhancing the accuracy and efficiency of intrusion detection in smart grid communication infrastructure. The goals of this research include:
\firmlist
\begin{itemize}
	\item Designing and implementing a deep learning intrusion detection system for security threats (DoS and DDoS) facing smart grid communication networks with LSTM and CNN methods.
	\item Evaluating the proposed system's performance using evaluation metrics such as accuracy, precision, recall, and F1-score.
\end{itemize}

 



\section*{Document Plan}
This thesis is organized as follows: In the first chapter, we provide an overview and the state of the art of the smart grid communication infrastructure and the importance of cybersecurity. In the second chapter. In the second chapter, we describe the proposed system and its components. And we present the experimental results and evaluation of the proposed system. Finally, in the conclusion, we conclude the thesis and discuss the implications of the results.